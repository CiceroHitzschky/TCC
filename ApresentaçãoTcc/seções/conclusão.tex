\AtBeginSection[]{
	\begin{frame}
		\frametitle{}
		\tableofcontents[currentsection]
	\end{frame}
}

\section{Conclusão}
	\begin{frame}{Conclusão}
		\begin{block}{Objetivos Específicos}
			\begin{itemize}[<+->]
				\item Definir a base do estudo da teoria da medida por meio dos espaços mensuráveis;
				\item Conhecer a teoria da medida de maneira generalizada;
				\item Descrever o processo da construção da integral de Lebesgue mediante o avanço da teoria da medida.
			\end{itemize}
		\end{block}
	\end{frame}

	\begin{frame}{Conclusão}
		\begin{block}<+->{Primeiro Objetivo Específico}
			\justify Definir a base do estudo da teoria da medida por meio dos espaços mensuráveis.
		\end{block}
		\begin{block}{Formas de Alcançá-lo}
			\begin{itemize}[<+->]
				\item Definimos uma $\sigma$-álgebra de um conjunto;
				\item Apresentamos exemplos com grau crescente de abstração;
				\item Enunciamos e Provamos resultados sobre o tema.
			\end{itemize}
		\end{block}
	\end{frame}

	\begin{frame}{Conclusão}
		\begin{block}{Segundo Objetivo Específico}
			 Conhecer a teoria da medida de maneira generalizada.
		\end{block}
	\begin{block}{Formas de Alcançá-lo}
		\begin{itemize}[<+->]
				\item Extensão da Reta Real;
				\item Generalizamos resultados apresentados anteriormente;
				\item Definição de Medida e exemplos;
				\item Divisão do Teorema em vários Resultados.
			\end{itemize}
		\end{block}
	\end{frame}


	\begin{frame}{Conclusão}
		\begin{block}{Último Objetivo Específico}
			\justify Descrever o processo da construção da integral de Lebesgue mediante o avanço da teoria da medida.
		\end{block}
		\begin{block}{Formas de Alcançá-lo}
			\begin{itemize}[<+->]
				\item Retomada da construção de Riemann;
				\item Explicitação de casos particulares até o caso geral;
				\item Exibição de Gráficos.
			\end{itemize}
		\end{block}
	\end{frame}















