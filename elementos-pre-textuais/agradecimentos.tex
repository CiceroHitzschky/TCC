A Deus, por todas as bênçãos que me concedeu nesta vida.\\
Aos meus professores de ensino médio Ellen Lima e Renato Castro pelo incentivo aos estudos.\\
Aos professores Leo Ivo da Silva Souza e Jose Eduardo Moura Garcez pelos conselhos e motivações acadêmicas.\\
Ao Prof. Dr. Nícolas Alcântara de Andrade por me apresentar esta maravilhosa teoria no curso de probabilidade e pelas sugestões neste trabalho. \\
Ao Prof. Dr. Claudemir Silvino Leandro pela exímia orientação deste trabalho.\\
A Banca Examinadora pelas dicas e sugestões do aprimoramento deste trabalho.\\
%Ao meu grande amigo Victor Júlio pelas correções gramaticais e ortográficas.\\
Ao Programa de Monitoria Acadêmica (PROMAC), que me possibilitou ser monitor da disciplina de Cálculo Diferencial e Integral I e III dando a oportunidade de aprofundar meus conhecimentos matemático e docênte.\\
Aos colegas da Universidade Estadual do Ceará (UECE) pelo companheirismo e ajuda nesta etapa da minha vida.\\




