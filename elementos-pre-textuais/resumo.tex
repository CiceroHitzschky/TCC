A teoria da medida e da integração é um tema importante para o avanço nos estudos de matemática.
Esta foi desenvolvida, inicialmente, por Bernhard Riemann (1826-1866), Georg Cantor (1845-1918) e Emile Borel (1871-1956) sendo generalizada posteriormente por Henri Lebesgue
(1875-1941).
A priori, seu desenvolvimento tinha o intuito de generalizar a integral de Riemann corrigindo o defeito de só valer para casos excepcionais, mas nos dias atuais possui aplicações nas mais diversas áreas tais como: Análise Funcional, Probabilidade, Estatística, Equações Diferenciais Parciais, dentre outras. 
Embora a teoria da integração de Lebesgue seja extremamente importante na atualidade não é comumente apresentada para alunos de graduação em ciências exatas.
Dito isso, este trabalho visa abordar a teoria da medida e da integração de Lebesgue de maneira introdutória para alunos de graduação em ciências da natureza buscando expôr alguns dos resultados mais relevantes e elementares.
Isso é feito utilizando uma metodologia de natureza básica com abordagem quantitativa através de uma revisão bibliográfica.
Por meio dela, também conseguimos alcançar os seguintes objetivos específicos: definir a base do estudo da teoria da medida por meio dos espaços mensuráveis,
conhecer a teoria da medida de maneira generalizada e descrever o processo da construção da integral de Lebesgue mediante o avanço da teoria da medida.
Por fim, é sugerido uma proposta de exposição do tema abordado neste trabalho para alunos de graduação em ciências exatas.

% Separe as palavras-chave por ponto
\palavraschave{teoria da medida; teoria da integração de Lebesgue; ensino de cálculo.}
