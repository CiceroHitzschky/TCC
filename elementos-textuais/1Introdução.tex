
%\textbf{Objetivo Geral} \\ \\
%Conhecer aplicações da convergência de sequências de números reais no Cálculo Diferencial e Integral.  \\ % partindo de diferentes formas de demonstrações.

%\textbf{Objetivos específicos} \\ \\
%a)	Relatar algumas circunstâncias da utilização de sequências sob contextos históricos; \\ \\
%b)	Apontar teoremas que qualificam a convergência de sequências de números reais, formulando suas respectivas demonstrações; \\ \\
%c) Apresentar algumas aplicações da convergência de sequências de números reais e suas possibilidades de demonstrações no Cálculo Diferencial e Integral.


%c) Apresentar algumas possibilidades de demonstrações para casos de aplicações da convergência de sequências de números reais no Cálculo Diferencial e Integral.

\chapter{Introdução}
    Para que a ação desejada tenha total eficácia é indispensável a projeção de um plano. 
    Como menciona Gondim et al (2010, p.41) 
    \enquote{[...] seu papel  semelhante àquele desempenhado pela planta de uma cidade, ou seja, é um guia básico para quem quer conhece-la ou si chegar ao seu destino com eficiência}
    Neste texto apresenta-se o Projeto de Trabalho de Conclusão de Curso do aluno de Licenciatura em Matemática pela Universidade Estadual do Ceará Cícero Moreira Hitzschky Filho.

    Dito isso, é apresentado neste projeto é apresentado o que levara ao autor realizar a escolha do tema, bem como suas motivações para tal.
    Adiante, o autor justifica esta escolha expondo a relevância do tema escolhido, bem como suas possíveis aplicações.
    Em seguida, ele faz uma fundamentação teórica sobre o tema e apresenta o que ele pretende na realização da sua pesquisa.

    Por fim, é apresentado os procedimentos bem como o passo a passo que o autor pretende fazer na realização de sua pesquisa.
    O cronograma desses procedimento também é apresentado ao passo que é apresentado uma possível estruturação de seu futuro trabalho.