
%\textbf{Objetivo Geral} \\ \\
%Conhecer aplicações da convergência de sequências de números reais no Cálculo Diferencial e Integral.  \\ % partindo de diferentes formas de demonstrações.

%\textbf{Objetivos específicos} \\ \\
%a)	Relatar algumas circunstâncias da utilização de sequências sob contextos históricos; \\ \\
%b)	Apontar teoremas que qualificam a convergência de sequências de números reais, formulando suas respectivas demonstrações; \\ \\
%c) Apresentar algumas aplicações da convergência de sequências de números reais e suas possibilidades de demonstrações no Cálculo Diferencial e Integral.


%c) Apresentar algumas possibilidades de demonstrações para casos de aplicações da convergência de sequências de números reais no Cálculo Diferencial e Integral.

\chapter{Introdução}
% Justificativa

    Conseguir medir grandezas é essencial em nosso cotidiano.
    Medimos tempo, distância, velocidade, etc.
    Embora seja usada constantemente nos dias atuais, a ação de medir é bem antiga possuindo registro de seu uso em 3100 a.C.
    Em relação à essa época, \citeauthor{boy}(\citeyear{boy}) expõe que os egípcios possuíam uma precisão grandiosa no que se referia a contar e medir;
    com essa habilidade, fizeram feitos extraordinários como o calendário solar e a construção das pirâmides.
    Ou seja, a ação de medir era dada, matematicamente, de forma empírica não havendo o famoso rigor matemático que conhecemos hoje.
   
    \enquote{
    	Pelo fim do século dezenove, a ênfase no rigor levou numerosos matemáticos à produção de exemplos de funções 'patológicas' que, devido a alguma propriedade incomum, violavam um teorema que antes se supunha válido em geral
    } \cite[p.415]{boy}.
    Isso gerou um grande movimento matemático, no século XX, que tinha como objetivo trazer um rigor unificado à matemática.
    Foi nesse movimento que o matemático francês Henri Lebesgue (1875-1941) notou aplicações limitadas da integral de Riemann através do seguinte pensamento:
    
    \begin{citlon}
    	Lebesgue, refletindo sobre o trabalho de Borel sobre conjuntos, viu que a definição de Riemann de integral tem o defeito de só se aplicar a casos
    	excepcionais, pois assume não mais que uns poucos pontos de descontinuidade para a função.
    	Se uma função $y = f(x)$ tem muitos pontos de descontinuidade, então, à medida que o intervalo $x_{i+1} - x_i$
    	se torna menor, os valores $f(x_{i+1})$ e $f(x_i)$ não ficam
    	necessariamente próximos \cite[p.416]{boy}.  	
    \end{citlon}

   % tema
    Para resolver este problema, Lebesgue decidiu inverter a ordem da construção que Riemann fizera:
    
    \begin{citlon}
    	Em vez de subdividir o domínio da variável independente, Lebesgue dividiu, portanto, o campo de variação $\overline{f} - f$ 
    	da função em subintervalos $\Delta y_i$ e em cada subintervalo
    	escolheu um valor $\eta_1$. 
    	Então, achou a \enquote{medida} $\mu(E_i)$ do conjunto $E_i$ dos pontos do eixo $x$ para os
    	quais os valores de $f$ são aproximadamente iguais a $\eta_1$\cite[p.416]{boy}.
    \end{citlon}
    
    A integral que foi gerada por meio da construção anterior recebeu o nome de Integral de Lebesgue que é o tema deste trabalho.
    
    % Problematização e Problema de Pesquisa
    Realizando uma pesquisa com a frase \enquote{ensino de cálculo diferencial e integral} na Biblioteca Digital Brasileira de Teses e Dissertações (BDTD), sem
    aspas, foram retornados 310 resultados na data do dia 29 de Outubro de 2023 às 10 horas da manhã.
    Ao observar título e resumo dos trabalhos retornados como resultado da busca, nenhum deles tratava da integral de Lebesgue para alunos de graduação em ciências exatas.
    Grande parte dos trabalhos eram voltados à reflexão do ensino de cálculo diferencial e integral na perspectiva de Riemann ou aplicações dela.
    Mediante esta ausência, esta pesquisa levanta o seguinte questionamento: de que forma pode ser apresentada a teoria da medida e integração de Lebesgue de maneira elementar para os alunos de graduação em ciências exatas?
    
    % Objetivos Gerais e Específicos
    Com base nisso, o objetivo geral desta pesquisa é conhecer algumas definições e resultados da teoria da medida e da integração de Lebesgue de forma elementar.
    Esse estudo estabeleceu três objetivos específicos: 
    definir a base do estudo da teoria da medida por meio dos espaços mensuráveis,
    conhecer a teoria da medida de maneira generalizada e descrever o processo da construção da integral de Lebesgue mediante o avanço da teoria da medida.
    
    % Prévias dos capítulos
  	
  	Na primeira seção definimos toda a base de nosso estudo tratando dos conceitos de $\sigma$-álgebra, espaços mensuráveis e funções mensuráveis.
  	Todos esses conceitos são vastamente explorados por meio de exemplos.
  	Em particular, é apresentado a $\sigma$-álgebra de Borel, bem como é revisado a estrutura de conjuntos abertos na reta numérica.
  	Também mostramos propriedades aritméticas e resultados sobre funções mensuráveis.
  	Por fim, a seção é finalizada com os conceitos de parte positiva e parte negativa de uma função real.
  	
  	Adiante apresentamos, na segunda seção, a teoria da medida. 
  	Para isso, iniciamos com uma generalização dos conceitos da seção anterior utilizando a reta real estendida.
  	Estendemos as definições e resultados de funções mensuráveis para valores reais estendidos.
  	Encerramos a seção apresentando o conceito geral de medida sobre um espaço mensurável apontando exemplos interdisciplinares tais como a função de probabilidade e a medida de um conjunto discreto.
  	
  	Na última seção, mostramos a construção da integral de Lebesgue para funções simples, funções mensuráveis não negativas e funções mensuráveis quaisquer, respectivamente.
  	Desenvolvemos propriedades e condições de integrabilidade ressaltando teoremas relevantes como o Teorema da Convergência Monótona e o Lema de Fatou.
  	Posteriormente abordamos um vasto número de proposições e propriedades sobre integrais com a finalidade de familiarização sendo a seção terminada com o Teorema da Convergência Dominada de Lebesgue.
  	
  
  	% Metodologia
  	
  	A pesquisa que foi realizada tem natureza básica, pois segundo \cite[p.51]{profreitas} uma pesquisa básica
  	\enquote{objetiva gerar conhecimentos novos úteis para o avanço da ciência sem aplicação prática prevista. Envolve verdades e interesses universais}. 
  	
  	Além disso, de acordo com o mesmo autor, a pesquisa também constou com uma abordagem quantitativa uma vez que os fatos podem ser relevados fora de uma complexidade social, econômica e política \cite{profreitas}. 
  	Pode-se observar, ainda, que a pesquisa teve objetivo exploratório, pois segundo \citeauthor{gil}:
  	
  	\begin{citlon}
  		As pesquisas exploratórias têm como propósito proporcionar maior familiaridade com o problema, com vistas a torná-lo mais explícito ou a construir hipóteses. 
  		Seu planejamento tende a ser bastante flexível, pois interessa considerar os mais variados aspectos relativos ao tato ou fenômeno estudado.
  		A coleta de dados pode ocorrer de diversas maneiras, mas geralmente envolve: 1.levantamento bibliográfico;
  		2. entrevistas com pessoas que tiveram experiência prática com o assunto; 
  		e 3. análise de exemplos que estimulem a compreensão (SELLTIZ et al., 1967, p. 63). 
  		Em virtude dessa flexibilidade, torna-se difícil, na maioria dos casos, "rotular" os estudos exploratórios, mas é possível identificar pesquisas bibliográficas, estudos de caso e mesmo levantamentos de campo que podem ser considerados estudos exploratórios (\citeyear{gil}, p.33).
  	\end{citlon}
  	
  	Como fora supracitado, pode-se atribuir vários procedimentos técnicos à pesquisas exploratórias. 
  	O nosso procedimento técnico foi realizado por meio de uma pesquisa bibliográfica pois foi elaborada com base em material já publicado \cite{gil}.
  	
  	Por fim, sugerimos como pode ser apresentada a teoria da medida e da integração de Lebesgue para alunos de graduação em ciências exatas.
  	Utilizando a ordem com que os conteúdos foram desenvolvidos e organizados neste trabalho, bem como o uso das ilustrações nele presentes.
    
    
    
    
    
    
    
    
    
    
    
    
    
    
    
    
    
    