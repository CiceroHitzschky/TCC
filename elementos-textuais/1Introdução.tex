
%\textbf{Objetivo Geral} \\ \\
%Conhecer aplicações da convergência de sequências de números reais no Cálculo Diferencial e Integral.  \\ % partindo de diferentes formas de demonstrações.

%\textbf{Objetivos específicos} \\ \\
%a)	Relatar algumas circunstâncias da utilização de sequências sob contextos históricos; \\ \\
%b)	Apontar teoremas que qualificam a convergência de sequências de números reais, formulando suas respectivas demonstrações; \\ \\
%c) Apresentar algumas aplicações da convergência de sequências de números reais e suas possibilidades de demonstrações no Cálculo Diferencial e Integral.


%c) Apresentar algumas possibilidades de demonstrações para casos de aplicações da convergência de sequências de números reais no Cálculo Diferencial e Integral.

\chapter{Introdução}
    Conseguir medir grandezas é essencial em nosso cotidiano.
    Medimos tempo, distância, velocidade, etc.
    Embora seja usada constantemente nos dias atuais, a ação de medir é bem antiga.
    Conforme Boyer (2012), os egípcios possuíam uma precisão grandiosa no que se referia a contar e medir;
    com essa habilidade, fizeram feitos extraordinários como o calendário solar e a construção das pirâmides.
    Desde então vem-se trabalhando no avanço da teoria da medida tanto de forma prática como de forma teórica.
    
    As aplicações mais teóricas de medida vieram com os estudos de Newton que trataram sobre taxa de variação de quantidades continuamente variáveis tais como
    comprimentos, áreas e volumes; tendo desenvolvido, também, um método para calcular a soma das ordenadas sob uma curva (Boyer, 2012). 
    Embasando-se no mesmo autor, vemos que esse método recebeu o nome de integral e foi denotado, por Leibniz, como $\int$; essa definição de integral foi refinada por Riemann e é, nos dias de hoje, comumente chamada de Integral de Riemann. 
    
    Pelo fim do século XIX, muitos matemáticos buscavam refutar teoremas já estabelecidos por meio de contraexemplos construídos por funções \enquote{patológicas}(Boyer, 2012).
    Isso gerou um grande movimento matemático, no século XX, que tinha como objetivo trazer rigor à matemática.
    Foi nesse movimento que o matemático francês Henri Lebesgue (1875-1941) notou aplicações limitadas da integral de Riemann através do seguinte pensamento:
    
    \begin{citlon}
    	Lebesgue, refletindo sobre o trabalho de Borel sobre conjuntos, viu que a definição de Riemann de integral tem o defeito de só se aplicar a casos
    	excepcionais, pois assume não mais que uns poucos pontos de descontinuidade para a função.
    	Se uma função $y = f(x)$ tem muitos pontos de descontinuidade, então, à medida que o intervalo $x_{i+1} - x_i$
    	se torna menor, os valores $f(x_{i+1})$ e $f(x_i)$ não ficam
    	necessariamente próximos (Boyer, 2012, p.416).  	
    \end{citlon}
    
    Para resolver este problema, Lebesgue decidiu inverter a ordem da construção que Riemann fizera:
    
    \begin{citlon}
    	Em vez de subdividir o domínio da variável independente, Lebesgue dividiu, portanto, o campo de variação $\overline{f} - f$ 
    	da função em subintervalos $\Delta y_i$ e em cada subintervalo
    	escolheu um valor $\eta_1$. 
    	Então, achou a \enquote{medida} $\mu(E_i)$ do conjunto $E_i$ dos pontos do eixo $x$ para os
    	quais os valores de $f$ são aproximadamente iguais a $\eta_1$(Boyer, 2012, p.416).
    \end{citlon}
    
    A integral que foi gerada por meio da construção anterior recebeu o nome de Integral de Lebesgue que é o tema deste trabalho.
    
    Realizando uma pesquisa com a frase \enquote{ensino de cálculo diferencial e integral} na Biblioteca Digital Brasileira de Teses e Dissertações (BDTD), sem
    aspas, foram retornados 310 resultados na data do dia 29 de Outubro de 2023 às 10 horas da manhã.
    Ao observar título e resumo dos trabalhos retornados como resultado da busca, nenhum deles tratava a integral de Lebesgue para alunos de graduação em ciências exatas.
    Grande parte dos trabalhos eram voltados à reflexão do ensino de cálculo diferencial e integral na perspectiva de Rienmann ou aplicações dela.
    Mediante esta ausência, surge o seguinte questionamento: de que forma pode ser apresentada a teoria da medida e integração de Lebesgue para alunos de graduação de ciências exatas?
    
    
    
    
    
    
    
    
    
    
    
    
    
    
    
    
    
    
    