\chapter{Alguns Aspectos Históricos}

\section{O surgimento da Análise}
\section{A Generalização da Integral de Riemann e a Integral de Lebesgue}
\section{A Consolidação da Teoria da Medida}
Com o intuito de identificar pesquisas realizadas sobre o tema, fora realizada uma busca na Biblioteca Digital brasileira de Teses e Dissertações (BDTD).
Nesta pesquisa, optou-se por pesquisar pelas palavras-chave \enquote{Teoria da Medida}, \enquote{Espaços $L^p$} e \enquote{Análise Funcional}. 
Houve uma delimitação de tempo de 5 anos para garantir que as pesquisas desenvolvidas fossem atuais. Assim, o resultados apresentados adiante foram realizados entre 2018 e 2023.

Ao pesquisar pelo termo "Teoria da Medida", entre aspas duplas, nos campos de Título, Assunto e Resumo em Português nenhum resultado foi encontrado. 
Mudando o Assunto para Matemática sem aspas e permanecendo os outros como estavam nenhum resultado fora encontrado novamente.
Depois disso, foram realizadas algumas variações da forma de pesquisa que apresentavam vários resultados. 
Esses resultados, por sua vez, não eram apenas matemáticos.
Por exemplo, ao pesquisar "Teoria da Medida" apenas no campo Assunto, retornou, dentre outros,  a dissertação de título 	
\enquote{Beyond psychometric assumptions : how to develop new psychological measures} que traduzido para o português se torna 
\enquote{Além das suposições psicométricas: como desenvolver novas medidas psicológicas}. 

Dessa forma, por não encontrar trabalhos relacionados ao primeiro termo, fora pesquisado o termo "Espaços $L^p$"  entre aspas, separando \enquote{espaços} de \enquote{lp} por meio de um sub-traço separando essas duas palavras pondo-se o mesmo período de tempo do anterior. 
Com isso, foram retornados 4 resultados. 
O primeiro deles é a dissertação de mestrado que tem por título \textit{A Desigualdade de Grothendieck e os Teoremas de Grothendieck para operadores absolutamente somantes definidos em espaços $L^p$,}. Esse trabalho apresenta o uso dos espaços $L^p$ de maneira relevante nos estudos da matemática atual ressaltando a importância do tema.
Embora vários resultados sejam apresentados o principal objetivo do traballho de Zagnoli de Assis (2019) é apresentar a desigualdade de Grothendieck.

O segundo, é uma tese de doutorado com o título \textit{Unicidade e Estabilidade de Hipersuperfícies em Espaços semi-Riemannianos} (OLIVEIRA, 2018, tradução nossa).
Este trabalho não fala diretamente de espaços $L^p$, mas depende totalmente desses espaços para ser realizado.
O foco do trabalho é todo voltado à hipersuperfícies verificando se suas funções simétricas apresentam algum tipo de estabilidade o que vai muito além do que se é visto em um curso de graduação, principalmente em licenciatura em matemática.

Outro trabalho que apareceu foi o intitulado \textit{Espaços $L^p$, Não-Comutativos e Perturbações de Estados KMS}.
Esse trabalho é a tese de doutorado de Silva (2018) no qual é realizada uma extensão da teoria das pertubações através de espaços $L^p$ não comutativos e novamente ultrapassa os conteúdos vistos na graduação sendo uma leitura muito difícil para alunos desta.

O último trabalho que apareceu em relação a esta palavra chave temos mais uma dissertação de mestrado com o título \textit{Uniform homeomorphisms between unit
spheres of interpolation spaces} onde Gesing (2020) faz um estudo detalhado sobre outro artigo de mesmo tema. 
Aqui vemos novamente o uso dos espaços $L^p$ de maneira direta para geometria por meio do estudo dos homeomorfismos.

Seguindo a pesquisa na BDTD, temos o termo de pesquisa que foi "Análise Funcional" que foi posta entre aspas por meio de dois campos de filtros de busca Assunto e Todos os Campos onde foram colocados, respectivamente, "analise\_funcional" e matemática tal qual estão escritas retornando um total de 28 trabalhos dentre dissertações de mestrado e teses de doutorado das mais variadas áreas e aplicações matemáticas.
Dentre esses resultados foram escolhidos os três nos quais apresentam os seguintes títulos: \textit{Espaços de operadores lineares, multilineares e polinômios regulares em espaços de Riesz e reticulados de Banach}; \textit{Multipolynomials between Banach spaces : designs of a theory}; \textit{Equação ao de Kirchhoff fracamente dissipativa: existência, unicidade e decaimento exponencial}.

Todos os trabalhos que vieram da palavra chave \enquote{Análise Funcional} tratam de matemática muito mais avançada do que a de graduação sendo muito dificultoso ou quase impossível um aluno convencional de graduação conseguir ler por tratar de teses de doutorado.
Isso também ocorreu com os trabalhos que vêm da palavra chave \enquote{Espaços $L^p$}. 
Nenhum deles apresenta trabalhos ou resultados elementares para que seja de fácil entendimento para alunos a nivel de graduação.

Desta forma, tendo em vista tudo que fora pesquisado, percebe-se, sem dificuldades, que há ausência de estudos elementares sobre matemática pura, particularmente falando de Espaços $L^p$. Diante disso, a pesquisa propor-se-á responder a seguinte pergunta norteadora: Como apresentar a teoria da medida por meio dos espaços $L^p$ exibindo aplicações na análise funcional de maneira elementar?