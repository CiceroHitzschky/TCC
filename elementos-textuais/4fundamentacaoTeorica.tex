\chapter{TEORIA DA MEDIDA E OS ESPAÇOS $L^p$}   
    \section{A Sistema Numérico da Reta Estendida}
    \section{Funções Mensuráveis}
        \subsection{Conjuntos e Funções Mensuráveis}
        \subsection{Espaços Mensuráveis}
        \subsection{Funções Entre Espaços Mensuráveis}
    \section{Medida}
        \subsection{Medidas}
        \subsection{Espaços de Medida}
    \section{Integração em Espaços de Medida}
        \subsection{Funções Simples e suas Integrais}
        \subsection{A Integral de uma Função Mensurável com Valores não Negativos na Reta Estendida}
        \subsection{Teoremas Importantes Sobre Integrais}
    \section{Funções Integráveis}
        \subsection{Integração de Funções de Valores Reais Estendidos}
        \subsection{Positividade e Linearidade da Integral}
    \section{Os Espaços de Lebesgue $L^p$}
        \subsection{Os espaços $L^p$}
        \subsection{Os espaços $L^\infty$}
    
    No século XIX, muitos teoremas que há muito eram utilizados para avanços da matemática começaram a receber questionamentos.
    De acordo com Boyer vemos que \enquote{Pelo fim do século dezenove, a ênfase no rigor levou numerosos matemáticos à produção de exemplos de funções “patológicas” que, devido a alguma propriedade incomum, violavam um teorema que antes se supunha válido em geral} (\citeyear{Boy}, p. 415).    
    
    Com o Cálculo Integral não foi diferente. 
    Embora a teoria da integral de Riemann seja consistente em um espaço euclidiano, $\R^n$ por exemplo, ao trabalharmos em espaços mais gerais alguns resultados não são válidos.
    Haja vista a necessidade de medir conjuntos mais gerais o matemático francês Henri Lebesgue (1875-1941) desenvolveu seu próprio conceito de integral que generalizava a integral de Riemann (BOYER, \citeyear{Boy}).

    Segundo Boyer (\citeyear{Boy}) Lebesgue viu que a integral construída por Rienmann possuía um defeito que era a quantidade de pontos de continuidade.
    Ou seja, para Lebesgue 
    \enquote{Se uma função $y = f(x)$ tem muitos pontos de descontinuidade, então, à medida que o intervalo $x_{i+1} - x_i$ se torna menor, os valores $f(x_{i+1})$ e $f(x_i)$ não ficam necessariamente próximos} (BOYER,\citeyear{Boy}, p.416).

    Ao estudarmos análise na reta podemos perceber que Lebesgue se referia ao seguinte teorema 
    \enquote{Para que uma função limitada $f: [a,b] \to \R$ seja integrável, é necessário e suficiente que o conjunto \textbf{D} dos seus pontos de descontinuidade tenha medida nula} (ELON, \citeyear{Elon}, p.344).
    Embora a integral de Lebesgue seja válida para espaços mais gerais, seu método de construção é análogo ao método realizado por Riemann.
    A diferença, segundo Boyer (\citeyear{Boy}, p.417), encontra-se que 
    
    \begin{citlon}
        Em vez de subdividir o domínio da variável independente, Lebesgue dividiu, portanto, o campo de variação $\overline{f} - f$ da função em subintervalos 
        $\Delta y_i$ e em cada subintervalo escolheu um valor $\eta_1$. 
        Então, achou a 'medida' $m(E_i)$ do conjunto $E_i$ dos pontos do eixo $x$ para os quais os valores de $f$ são aproximadamente iguais a $\eta_1$.
    \end{citlon}
    
    Segundo Boyer (\citeyear{Boy}), esta genialidade de Lebesgue na construção de sua integral faz com que a forma de medir seja extremamente eficaz tornando-a utilizada até os dias de hoje.
 
 
 
 
 
 
 
 
 