\chapter{OBJETIVOS}
\section{OBJETIVO GERAL}
    \begin{itemize}
        %\item  Apresentar a teoria da medida por meio dos espaços $L^p$ exibindo aplicações na análise funcional de maneira elementar.
        \item Introduzir a teoria da medida por meio dos espaços $L^p$ destacando aplicações relevantes e elementares na análise funcional.
    \end{itemize}
\section{OBJETIVOS ESPECÍFICOS}
    \begin{itemize}
        \item Conhecer a relevância da Análise para estudos de matemática pura;
        \item Fundamentar conceitos prévios sobre teoria da medida e análise funcional;
        \item Conhecer definições e propriedades dos espaços $L^p$;
        \item Empregar aplicações do estudo dos espaços $L^p$ na análise funcional.
    \end{itemize}

\begin{comment}

Com o intuito de dar resposta à pergunta diretriz estabelecida, essa pesquisa possui, como objetivo geral, introduzir a teoria da medida por meio dos espaços $L^p$ destacando aplicações relevantes e elementares na análise funcional. 
Isso acarreta  na estipulação de três objetivos específicos, apresentados adiante.

O primeiro objetivo especifico é conhecer a relevância da Análise para estudos de matemática pura. 
Isso será feito por meio de um percurso histórico.
Como segundo objetivo específico planeja-se fundamentar conceitos prévios sobre teoria da medida e análise funcional, pois não é um tema comum de graduação e  deseja-se que todo todos os requisitos para entender o texto estejam contidos no próprio texto.  

O terceiro objetivo específico é o de conhecer definições e propriedades dos espaços $L^p$ e desta forma introduzir a teoria da medida.
Por fim, como quarto objetivo especifico, empregar aplicações do estudo dos espaços $L^p$ na análise funcional com a finalidade de correlacionar teoria da medida com a análise funcional.

\end{comment}
