
%\textbf{Objetivo Geral} \\ \\
%Conhecer aplicações da convergência de sequências de números reais no Cálculo Diferencial e Integral.  \\ % partindo de diferentes formas de demonstrações.

%\textbf{Objetivos específicos} \\ \\
%a)	Relatar algumas circunstâncias da utilização de sequências sob contextos históricos; \\ \\
%b)	Apontar teoremas que qualificam a convergência de sequências de números reais, formulando suas respectivas demonstrações; \\ \\
%c) Apresentar algumas aplicações da convergência de sequências de números reais e suas possibilidades de demonstrações no Cálculo Diferencial e Integral.


%c) Apresentar algumas possibilidades de demonstrações para casos de aplicações da convergência de sequências de números reais no Cálculo Diferencial e Integral.

\chapter{Considerações Finais}
	Neste trabalho abordamos a teoria da medida e integração, em sua forma elementar, tendo como pergunta diretriz de que forma pode ser apresentada a teoria da medida e integração de Lebesgue de maneira elementar para os alunos de graduação de ciências exatas?
	Com o intuito de auxiliar na obtenção da resposta, nosso objetivo geral que nos guiou em todo processo de construção do objeto de pesquisa foi: 
	conhecer a algumas definições e resultados da teoria da medida e da integração de Lebesgue de forma elementar.
	
	Desta forma, nosso primeiro objetivo específico era: definir a base do estudo da teoria da medida por meio dos espaços mensuráveis.
	Para atingir esse objetivo, primeiro definimos  uma \sigal de um conjunto.
	Em seguida, mostramos que um conjunto não vazio e uma \sigal desse conjunto formam um espaço mensurável.
	Fixamos este conceito apresentando exemplos com grau crescente de abstração.
	Provamos algumas proposições sobre espaços mensuráveis e apresentamos uma \sigal especial no conjunto dos números reais, a \sigal de Borel.
	Fizemos um processo semelhante para expor as funções mensuráveis sendo os resultados focados em uma maneira mais fácil de averiguar se uma função é mensurável sem utilizar a definição diretamente.
	
	Nosso segundo objetivo específico era: conhecer a teoria da medida de maneira generalizada.
	Para obter com sucesso esse objetivo, precisou-se estender a reta real adicionando as noções de $-\infty$ e $+\infty$ exibindo uma álgebra específica a ser adotada.
	Com isso, na primeira parte da seção, generalizamos os resultados apresentados na seção anterior dando ênfase às mudanças que ocorreram ao mudar de $\R$ para $\overline{\R}$.
	Depois disso, definimos o que é uma medida, enunciamos e provamos propriedades sobre.
	Além disso, trouxemos vários exemplos, dentre eles, o espaço de probabilidades e a medida de um conjunto discreto.
	Todas as demonstrações tiveram o intuito de serem as mais didáticas possíveis sendo os teoremas divididos em várias proposições para que pudesse haver o melhor entendimento.
	
	Nosso último objetivo era: descrever o processo da construção da integral de Lebesgue mediante o avanço da teoria da medida.
	Para tal, desmembramos os tipos de funções para que pudesse ser um processo intuitivo da construção.
	Exibimos, inicialmente, as integrais de Lebesgue para funções simples.
	Fixamos a medida de Lebesgue para que os exemplos ficassem mais claros, pois nesta parte, a diferença entre a integral de Lebesgue e a integral de Riemann era indistinguível. 
	Ao dar continuidade, lembramos a construção da integral de Riemann para funções não negativas para que depois pudéssemos mostrar a construção de Lebesgue e evidenciar a diferença entre as duas.
	Neste momento, várias ferramentas para integrais de Lebesgue para funções não negativas foram enunciadas e provadas para que esta pudesse ser generalizada para funções quaisquer.  
   	
   	Com isso, acreditamos que a forma como foi organizado este trabalho possibilita um estudo introdutório sobre a teoria da medida e integração de Lebesgue para alunos de graduação em ciências exatas em uma disciplina optativa.
   	Essa crença surge da simplificação do texto e da teoria para que aborde coisas pontais e não tão gerais como se é esperado em um primeiro contato com a teoria. 
   	Fazemos isso, por exemplo, quando decidimos começar diretamente com a definição de $\sigma$-álgebra ao invés de iniciar com a definição de semi-anel como faz o livro Medida e Integração de Pedro Jesus Fernandez (\citeyear{pedro}), por exemplo.
   	Além disso, tratamos primeiro de funções mensuráveis para, em seguida, trabalhamos com medida o que também não é comum na bibliografia brasileira.
   	Por mim, tentamos sempre por gráficos quando julgamos necessário o que também é escasso em livros sobre o tema por pensar em um público diferente ao que destina esse trabalho. 
   	 
    
    
    
    
    
    
    
    
    
    
    
    
    
    
    
    
    
    