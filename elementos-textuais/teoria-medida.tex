\chapter{A Teoria da Medida}
%%%%%%%%% ESPAÇOS MENSURÁVEIS
\section{Os Espaços de Funções Mensuráveis}

    As vezes, teremos conjuntos \enquote{grandes} de tal forma que sua medida não poderá ser expressa por um número real.
    Por exemplo, o conjunto $\R$.
    Por ser infinito 
    \footnote{Vide \supercite{elon}{p.86}},
    consideramos que seu tamanho é \enquote{infinito} e representamos pelo simbolo $+\infty$.
    Entretanto, $+\infty$ não é um número real e sim uma conceito de \enquote{tão grande quanto se queira}.
    Para solucionar esta problemática, vamos considerar um novo sistema numérico.

    \begin{definition}
    \label{def:reta-estendida}
        A coleção $\xreta$ que consiste de $\R \cup \{-\infty, +\infty\}$ é chamada de \textbf{Sistema Estendido de Números Reais}.
    \end{definition}

    Ou seja, $\xreta$ nada mais é que o conjunto dos números reais com a possibilidade de se ter $-\infty$ ou $+\infty$.
    Com isso, parece que nosso problema de medir conjuntos muito grandes se resolveu.
    Entretanto, alguns cuidados são necessários para operarmos em $\xreta$.
    Um deles, por exemplo, é que $\xreta$ não é fechado para operações de $\R$ tais como $(+\infty) + (-\infty)$ que nem definido é.
    Dito isso, para $x \in \R$, as operações dos símbolos $+\infty$ e $-\infty$ são dadas da seguinte forma:
    \begin{multicols}{2}
        \begin{itemize}
            \item $(+ \infty) + (+ \infty)  = + \infty$;
            \item $x + (+ \infty) = (+ \infty) + x = + \infty$;
            \item $(- \infty) + (- \infty)  = - \infty$;
            \item $x + (- \infty) = (- \infty) + x = - \infty$;
            \item $(+ \infty)\cdot (+ \infty) =  +\infty $;
            \item $(- \infty)\cdot (- \infty) =  +\infty $;
            \item $(+ \infty)\cdot (- \infty) =  -\infty $;
            \item $(- \infty)\cdot (+ \infty) =  -\infty $.
        \end{itemize}
    \end{multicols}
    Na multiplicação, dependendo do número real, a operação diferencia-se.
    Assim, podemos ter
    \begin{multicols}{2}
    $$
    x \cdot (+\infty) = (+\infty) \cdot x =
    \left\{\begin{array}{cc}
          +\infty, & \ \textrm{se } x > 0\\
          0, & \ \textrm{se } x = 0\\
          - \infty, & \textrm{se } x < 0
    \end{array}\right.
    $$
    
    $$
    x \cdot (-\infty) = (-\infty) \cdot x =
    \left\{\begin{array}{cc}
          -\infty, & \ \textrm{se } x > 0\\
          0, & \ \textrm{se } x = 0\\
          + \infty, & \textrm{se } x < 0
    \end{array}\right.
    $$  
    \end{multicols}

    Neste novo contexto de números reais a \sigal de Borel não é mais válida uma vez que a definição \ref{def:algebra-borel} não inclui $+\infty$ nem $-\infty$.
    Com isso, considere $\xreta$.
    Tomando um conjunto arbitrário $E \in \borel$, com $\varnothing \neq E$, defina $E_1 = E \cup \{-\infty\}, E_2 = E \cup \{+\infty\}$ e $E_3 = E \cup \{-\infty, +\infty\}$. 
    Desta forma, o conjunto $\overline{\borel} = \displaystyle \bigcup_{E \in \borel} \{E, E_1, E_2, E_3\}$ é uma \sigal de $\xreta$. Com efeito, se $E \in \borel$, então é um intervalo aberto conforme o teorema \ref{teo:equiv-borel}.
    Assim, $E_1, E_2, E_3$ e $E_4$ serão intervalos do tipo $[-\infty,x)$ ou $(x, +\infty]$ que são elementos de $\borel$ acrescidos de $+\infty$ ou $-\infty$. 
    Deste modo, é fácil verificar que se um elemento $A \in \xborel$, então $A^c \in \xborel$.
    Além disso, a união enumerável é, no máximo, o intervalo $[-\infty,+\infty]$ que é exatamente $\xreta$.
    Desta forma, $\xborel$ é uma \sigal de $\xreta$.

    \begin{definition}
    \label{def:algebra-borel-estendida}
        A \sigal $\overline{\borel} = \displaystyle \bigcup_{E \in \borel} \{E, E_1, E_2, E_3\}$ do conjunto $\xreta$ é chamada de Álgebra de Borel Estendida. 
    \end{definition}

    Uma vez que estamos familiarizados com os conceitos de funções de valores reais mensuráveis, estamos prontos para estender este conceito para o conjunto $\xreta$.

    \begin{definition}
    \label{def:familia-funcoes-mensuraveis}
        Sendo $(X, \mathcal{C})$ um espaço mensurável, uma função de valores reais estendidos $f: X \to \xreta$ é dita $\mathcal{C}$-mensurável caso o conjunto
        $\{x \in X; f(x) > \alpha\} \in \mathcal{C}$ para qualquer que seja $\alpha \in \R$. 
    \end{definition}

	Denotaremos a família de todas as funções de valores reais estendidos de $X$ que são $\mathcal{C}$-mensuráveis por $M(X, \mathcal{C})$.
    \begin{proposition}
    \label{prop:identidade-intersecao-mais-infinito}
        Se $f \in \menfus$, então $\{x \in X; f(x) = +\infty\} = \displaystyle \bigcap_{n = 1}^\infty \{x \in X; f(x) > n\}$.
    \end{proposition}

    \begin{prova}
        Tome, de modo arbitrário, um elemento $a \in X$. 
        Assim, 
        \begin{align*}
            a \in \bigcap_{n = 1}^\infty \{x \in X; f(x) > n\} 
            \Leftrightarrow & a \in \{x \in X; f(x) > n\}, \ \forall n \in \N\\
            \Leftrightarrow & \forall n \in \N,\ f(a) > n\\
            \Leftrightarrow & \dlim_{n \to \infty} f(a) \geq \dlim_{n \to \infty} n\\
            \Leftrightarrow & f(a) \geq +\infty.  
        \end{align*}
    Como estamos trabalhando com $\xreta$, não existe elemento $x > +\infty$.
    Logo, o único elemento possível para $f(a)$ é $+\infty$.
    Assim, tudo isso ocorre se, e somente se, o elemento $a \in \{x \in X; f(x) = +\infty\}$ como queríamos.
    Além disso, note que cada $\{x \in X; f(x) > n\} \in \mathcal{C}$.
    Segue, pela proposição \ref{prop:interseção-elementos-sigmas}, que $\displaystyle \bigcap_{n = 1}^\infty \{x \in X; f(x) > n\} \in \mathcal{C}$ acarretando que $\{x \in X; f(x) = +\infty\} \in \mathcal{C}$. 
    \end{prova}

    \begin{proposition}
    \label{prop:identidade-união-menos-infinito}
        Se $f \in \menfus$, então $\{x \in X; f(x) = -\infty\} = \displaystyle \left(\bigcup_{n = 1}^\infty \{x \in X; f(x) > - n\}\right)^c$.
    \end{proposition}
	%
    \begin{prova}
        Analogamente à proposição \ref{prop:identidade-intersecao-mais-infinito} tomemos $a \in X$. 
        Segue que 
        \begin{align*}
            a \in \left(\bigcup_{n = 1}^\infty \{x \in X; f(x) > - n\}\right)^c
            \Leftrightarrow & a \in \bigcap_{n = 1}^\infty \left(\{x \in X; f(x) > - n\}\right)^c\\
            \Leftrightarrow & \forall \ n \in \N, a \in \left(\{x \in X; f(x) > - n\}\right)^c\\
            \Leftrightarrow & \forall \ n \in \N, a \notin \{x \in X; f(x) > - n\}\\
            \Leftrightarrow & \forall \ n \in \N, a \in \{x \in X; f(x) \leq - n\}\\
            \Leftrightarrow & \forall \ n \in \N, f(a) \leq - n\\
            \Leftrightarrow & \lim_{n \to \infty} f(a) \leq \lim_{n \to \infty} (- n)\\            
            \Leftrightarrow & f(a) \leq -\infty\\            
            \Leftrightarrow & f(a) = -\infty\\            
            \Leftrightarrow & a \in \{x \in X; f(x) = -\infty\}            
        \end{align*}
        
    Ora, cada $\{x \in X; f(x) > - n\} \in \mathcal{C}$.
    Assim, por definição de \sigal, temos que $ \displaystyle\bigcup_{n = 1}^\infty \{x \in X; f(x) > - n\} \in \mathcal{C}$ e também 
    $\displaystyle\left(\bigcup_{n = 1}^\infty \{x \in X; f(x) > - n\}\right)^c \in \mathcal{C}$.
    Concluímos disso que $\{x \in X; f(x) = -\infty\} \in \mathcal{C}$ como queríamos provar. 
    \end{prova}

    \begin{theorem}
    \label{teo:condição-de-mensurabilidade}
        Uma função de valores reais estendidos $f: X \to \xreta$ é $\cc$-mensurável se, e somente se, os conjuntos 
        $A = \{ x \in X; f(x) = +\infty\}$ e $B = \{x \in X; f(x) = -\infty\}$
		 são elementos de $\mathcal{C}$ e a função $h: X \to \R$ definida por
		 
		 $$
		 h(x) = \left\{\begin{array}{cc}
		     f(x), & \textrm{\ se } x \notin A\cup B  \\
		      0,& \textrm{\ se } x \in A\cup B
		 \end{array}\right.
		 $$
		 é $\cc$-mensurável.
	 \end{theorem}
\begin{prova}
    Suponha que $f \in \menfus$. 
    Logo, pelas proposições \ref{prop:identidade-intersecao-mais-infinito} e \ref{prop:identidade-união-menos-infinito}, os conjuntos $A$ e $B$ são elementos de $\mathcal{C}$.
    Assim, tome $\alpha \in \R$ com $\alpha \geq 0$, então os elementos de $\{x \in X; h(x) > \alpha\}$ são os elementos de $\{x \in X; f(x) > \alpha\}$ que não estão em $A$, pois $h$ tem contradomínio $\R$.
    Como $\cc$ é uma \sigal, $A \in \cc \Rightarrow A^c \in \cc$. 
    Com isso, 
    $$
    \{x \in X; h(x) > \alpha\} = A^c\cap \{x \in X; f(x) > \alpha\} \in \cc
    $$
    Segue, pela proposição \ref{prop:interseção-elementos-sigmas} que $\{x \in X; h(x) > \alpha\} \in \cc$, ou seja, $h$ é $\cc$-mensurável.
    Caso, $\alpha < 0$, então $\{x \in X; h(x) > \alpha\} = \{x \in  X ; f(x) > \alpha\} \cup B $, pois $h(x) = 0$ para $x \in A \cup B$.
    Desta forma $h$ é $\cc$-mensurável.

    Por outro lado, se supormos que $A$ e $B$ são elementos de $\mathcal{C}$ e $h$ é $\cc$-mensurável, então
    $$\{x \in X; f(x) > \alpha\} = \{x \in  X ; h(x) > \alpha\} \cup A $$
    quando $\alpha \geq 0$, e 
    $$\{x \in X; f(x) > \alpha\} = \{x \in  X ; h(x) > \alpha\} - B $$
    quando  $\alpha < 0$ por motivos análogos à primeira parte da demonstração.
    Portanto, $f$ é uma função $\cc$-mensurável como desejávamos.
\end{prova}

Como consequência do teorema \ref{prop:aritmetica-uma-funcao} e o teorema \ref{teo:condição-de-mensurabilidade} obtemos, imediatamente, que se $ f \in M(X,\mathcal{C})$, então as funções $cf, f^2, |f|, f^+$ e $f^-$ também são elementos de $M(X, \mathcal{C})$.
Entretanto, um resultado análogo à proposição \ref{prop:aritmetica-duas-funcoes} não possível em $\xreta$.
Isso acontece porquê em $\xreta$ a operação de adição não é bem definida.
Então caso $f(x) = +\infty$ e $g(x) = -\infty$ para algum $x \in \R$ a adição
$f(x) + g(x)$ não é realizada.
Por outro lado, a função $fg$ é $\cc$-mensurável se $f$ e $g$ forem ambas $\cc$-mensuráveis.
Para mostrar isso, precisamos do seguinte teorema

\begin{theorem}
\label{teo:mensurabilidade-sequencia-funcoes-mensuraveis}
	Seja $(f_n)$ uma sequência de elementos de $\menfus$ e defina as funções
	$$f(x) = \inf f_n(x),\  
	F(x) = \sup f_n(x),\  
	f^*(x) = \lim\inf f_n(x),\   
	F^*(x) = \lim\sup f_n(x).$$
	Então as funções $f, f^*, F$ e $F^*$ são elementos de $\menfus$.
\end{theorem}
\begin{prova}
	Como $(f_n)$ é uma sequência de funções $\cc$-mensuráveis e $f = \inf f_n$, afirmamos que $\{x \in ; f(x) \geq \alpha\} = \displaystyle \bigcap_{n = 1}^\infty \{x \in X; f_n(x) \geq \alpha\}$.
	De fato, tomemos um elemento $h \in X$.
	Assim, 
		\begin{align*}
			h \in \bigcap_{n = 1}^\infty \{x \in X; f_n(x) \geq \alpha\} \Leftrightarrow	&
			h \in \{x \in X; f_n(x) \geq \alpha\}\ \forall n \in \N\\
			\Leftrightarrow	&	
			f_n(h) \geq \alpha\ \forall n \in \N\\
			\Leftrightarrow	&	
			\inf_{n \in \N}f_n(h) \geq \inf_{n \in \N}\alpha\ \forall n \in \N\\
			\Leftrightarrow	&	
			f(h) \geq \alpha\ \forall n \in \N\\
			\Leftrightarrow	&
			h \in \{x \in X; f(x) \geq \alpha\}
		\end{align*}
	Como cada $\{x \in X; f_n(x) \geq \alpha\}$ é $\cc$-mensurável, segue pela proposição \ref{prop:interseção-elementos-sigmas} que o conjunto $\{x \in X; f(x) \geq \alpha\} \in \cc$ para todo $\alpha \in \R$.
	Desta forma, $f$ é $\cc$-mensurável.
	
	Observe, também, que $\{x \in X; F(x) >\alpha\} = \displaystyle \bigcup_{n = 1}^\infty \{x \in X; f_(x) >\alpha\}$.
	Com efeito, para $h \in X$ 
	\begin{align*}
		h \in \bigcup_{n = 1}^\infty \{x \in X; f_n(x) > \alpha\} \Leftrightarrow	&
		\exists\ k \in \N \textrm{tal que\ } h \in \{x \in X; f_k(x) > \alpha\}\\
		\Leftrightarrow	&	
		f_k(h) > \alpha,\ \forall \alpha \in \R\\
		\Leftrightarrow	&	
		F(x) \geq f_k(h) > \alpha,\ \forall \alpha \in \R\\
		\Leftrightarrow	&	
		F(x) > \alpha,\ \forall \alpha \in \R\\
		\Leftrightarrow	&
		h \in \{x \in X; F(x) > \alpha\}
	\end{align*}
	Assim, concluímos que $f$ e $F$ são $\cc$-mensuráveis. 
	Note que a mensurabilidade de $f^*$ e $F^*$ vem de $f$ e $F$ uma vez que
	$$
	f^*(x) = \sup_{n \geq 1} \left\{\inf_{m \geq n} f_m(x)\right\}
	\textrm{\ e\ }
	F^*(x) = \inf_{n \geq 1} \left\{\sup_{m \geq n} f_m(x)\right\}
	$$
\end{prova}

\begin{corollary}
	\label{cor:convergencia-de-uma-sequencia-mensuravel}
	Se $(f_n)$ é uma sequência em $\menfus$ que converge para $f$ em $X$, então
	$f$ também está em $\menfus$.
\end{corollary}
\begin{prova}
	Ora, por hipótese $\displaystyle f(x) = \lim_{n \to +\infty} f_n(x)$.
	Só que $\displaystyle \lim_{n \to +\infty} f_n(x) = \lim_{n \in \N} \inf f_n(x)$.
	Segue que $\displaystyle f(x) = \lim_{n \in \N} \inf f_n(x)$ que, por sua vez, é $\cc$-mensurável pelo teorema anterior.
\end{prova}

% Parte final do Capítulo - Truncamento
\begin{definition}[Truncamento de uma função mensurável]
	Seja $f$ uma função em $\menfus$ e $A > 0$.
	Definimos o truncamento $f_A$ da função $f$ por
	$$ f_A(x) =
	\left\{\begin{array}{cc}
		f(x), & \textrm{se\ } |f(x)| \leq A \\
		A, & \textrm{se\ } f(x) > A \\
		-A, & \textrm{se\ } f(x) < A 
	\end{array}\right.
	$$
\end{definition}

% Exemplos de Truncamento

\begin{example}
	Seja $f \in \menfus$ tal que $f(x) = x^2-2$.
	Então o truncamento $f_2$ é representado, graficamente, como
	\begin{figure}[h!]
		\centering
		\Caption{\label{fig:representação do truncamento da função f(x) = x^2-2} representação do truncamento $f_2$ da função $f(x) = x^2-2$} 
		\UECEfig{}{
			\begin{tikzpicture}[scale=0.5]
				% Defina o intervalo x
				\def\xmin{-3}
				\def\xmax{3}
%				
%				% Desenhe a função f(x)
%				\draw[domain=\xmin+0.4:\xmax-0.4, smooth, samples=100, blue] plot (\x, {\x*\x -2});
				
				% Desenhando a função f_2
				\draw[domain=2:4.5, thick, samples=100, red] plot (\x, {2});
				\draw[domain=-4.5:-2, thick, samples=100, red] plot (\x, {2});
				\draw[domain=-2:2, thick, samples=100, red] plot (\x, {\x*\x -2});
				
				% Adicione rótulos aos eixos
				\draw[->] (\xmin-2,0) -- (\xmax+2,0) node[right] {$x$};
				\draw[->] (0,\xmin-1) -- (0,\xmax+2.5) node[above] {$y$};
				
                % Rótulos
				\foreach \i in {-4,-3,-2,-1,1,2,3,4}{
					\draw (\i,2pt)--(\i, -2pt) node[below]{{\footnotesize $\i$}};
				}
				
				\foreach \i in {1,2,3,4,5}{
					\draw (2pt,\i)--(-2pt, \i) node[left]{{\footnotesize $\i$}};
				}
			\end{tikzpicture}
		}{
			\Fonte{Elaborado pelo autor}		}   
	\end{figure}
	
\end{example}

	Observe que a figura \ref{fig:representação do truncamento da função f(x) = x^2-2} mostra que o truncamento $f_2$ efetua uma espécie de \enquote{limitação} da função $f$ pela constante $2$.

\begin{proposition}
	\label{prop:truncamento-mensurável}
	Seja $A$ um número real maior que zero.
	Se $f$ é uma função em $\menfus$, então $f_A$ é uma função $\cc$-mensurável.
\end{proposition}
\begin{prova}
	De fato, se os elementos $x \in X$ são tais que $-A \leq f(x) \leq A$, então $f_A(x) = f(x)$.
	Logo $f_A$ é $\cc$-mensurável, pois $f$ o é.
	Caso esses elementos sejam tais que $f(x) > A$ ou $f(x) <A$ a função $f_A$ é constante.
	Segue pela proposição \ref{ex:funcao-constante} que $f_A$ é um elemento de $\menfus$.
\end{prova}

Retornemos para a mensurabilidade do produto de duas funções com valores reais estendidos.
Sejam $f,g \in \menfus$. 
Tomemos duas sequências $(f_n)$ e $(g_m)$ tais que para cada $k \in \N$, $f_k$ e $g_k$ são truncamentos de $f$ e $g$, respectivamente.
Ou seja, 
	$$ g_m(x) =
		\left\{\begin{array}{cc}
			g(x), & \textrm{se\ } |g(x)| \leq m \\
			n, & \textrm{se\ } g(x) > m \\
			-n, & \textrm{se\ } g(x) < m 
		\end{array}\right.	
	$$
e $f_n$ é definida de modo similar.
Pela proposição \ref{prop:truncamento-mensurável}, $f_n$ e $g_m$ são $\cc$-mensuráveis para cada $n$ e $m$ números naturais.
Assim, pela proposição \ref{prop:aritmetica-duas-funcoes} $f_ng_m$ também é $\cc$-mensurável para quaisquer $n, m \in \N$.
Como o truncamento de uma função $f$ causa uma \enquote{limitação} na função $f$ se tomarmos $n$ grande suficiente o truncamento $f_n$ tende a se aproximar da função $f$.
Assim, para $x \in X$
$$
\lim_{n \to +\infty} \left(f_n(x)g_m(x)\right) = f(x)g_m(x) 
$$
Segue pelo corolário \ref{cor:convergencia-de-uma-sequencia-mensuravel} que 
$fg_m \in \menfus$. Com isso, temos que para $x \in X$
$$
\lim_{m \to +\infty} \left(f(x)g_m(x)\right) 
= f(x)g(x) = (fg)(x)
$$
Segue, pelo mesmo corolário, que $fg \in \menfus$.

Nos definimos a mensurabilidade de funções de maneira bem peculiar aos números reais uma vez que será o enfoque de nosso trabalho.
Entretanto, em alguns casos, é necessário trabalhar com mensurabilidade de uma forma mais abstrata.
Dito isso, encerraremos esta seção apresentando a definição generalizada de mensurabilidade de uma função.

\begin{definition}
	\label{def:mensurabilidade-geral}
	Sejam $(X, \cc)$ e $(Y,\mathcal{F})$ dois espaços mensuráveis.
	Dizemos que uma função $\phi:(X, \cc)\to (Y,\mathcal{F})$ é dita mensurável se o conjunto $f^{-1}(E) = \{x \in X; f(x)\in E\} \in \cc$ para todo conjunto $E \in \mathcal{F}$. 
\end{definition}

Embora essa definição pareça ser totalmente distinta da definição \ref{def:mensurabilidade-funções-reais}, as duas são equivalentes no caso particular de $Y = \R$ e $\mathcal{F} = \borel$ conforme demonstrado a seguir.

\begin{proposition}
	Seja $(X, \cc)$ um espaço mensurável e $f$ uma função.
	Então $f$ é $\cc$-mensurável se, e somente se. $f^{-1}(E) \in \cc$ para todo boreliano $E$. 
\end{proposition}
\begin{prova}
	Suponha $f$ uma função $\cc$-mensurável. 
	Sabemos pela definição \ref{def:algebra-borel} que os elementos da álgebra de Borel são do tipo $(-\infty,x)$ com $x \in \R$.
	Assim, dado arbitrariamente $\alpha \in \R$ temos que
	$$
	f^{-1}(-\infty, \alpha)
	=\{x \in X; f(x) \in (-\infty, \alpha)\}
	=\{x \in X; f(x) \leq \alpha\}
	%\footnote{$f^{-1}(-\infty, \alpha)$ indica a pré imagem de $(-\infty, \alpha)$.}
	$$
	Como $f$ é $\cc$-mensurável segue pelo teorema \ref{teo:equiv-funcoes-mensuraveis} que $f^{-1}(-\infty, \alpha) \in \cc$.
	Reciprocamente se 
	$f^{-1}(-\infty, \alpha) \in \cc$ para qualquer $\alpha$ concluímos, imediatamente, que $\{x \in X; f(x) < \alpha\} \in \cc$ para todo $\alpha \in \R$.
	Portanto, $f$ é $\cc$-mensurável.
\end{prova}
%%%%%%%%%% Espaços de Medida

\section{Espaços de Medida}

Nas subseções anteriores, nós trabalhos com conjuntos e com funções mensuráveis, isto é, que podem ser medidas de alguma forma.
Nesta subseção, nos preocuparemos em definir e trabalhar com funções de um espaço mensurável $(X, \mathcal{C})$ que daremos o nome de "medida".
Tais funções são induzidas pela nossa concepção de comprimento, área, volume, etc.
Dito isso, para trabalhamos com medidas primeiro retomaremos alguns resutlados sobre sequência de conjuntos.

\begin{definition}
\label{def:sequência-crescente-decrescente-de-conjuntos}
    Uma sequência de conjuntos $(A_n)$ é dita \textbf{crescente} se $A_n \subseteq A_{n+1}$ para todo $n \in \N$.
    Caso tenhamos $A_n \supseteq A_{n+1}$ para todo $n \in \N$, dizemos que a sequência  de conjuntos é \textbf{decrescente}.
\end{definition}

\begin{proposition}
\label{prop:sequencia-crescente-conjuntos-resultado-A_n}
Seja $(E_n)$ uma sequência crescente de conjuntos. Se $(A_n)$ é tal que $A_1 = E_1$ e $A_n = E_n - E_{n -1}$ para todo $n > 1$, então:
\begin{enumerate}[label* = (\roman*)]
    \item $A_n$ é uma sequência disjunta;
        \footnote{Lembre que uma sequência disjunta significa que $A_i \cap A_j = \varnothing$ para todo $i \neq j$}
    \item $E_n = \displaystyle \bigcup_{j = 1}^n A_n$;
    \item $\displaystyle \bigcup_{j = 1}^\infty E_n = \displaystyle \bigcup_{j = 1}^\infty A_n$;
\end{enumerate}
\end{proposition}

\begin{prova}
    Para provar $(a)$ precisamos mostrar que para todo $n,m \in \N$ se $m \neq n$, então $A_n \cap A_m = \varnothing$.
    Lembre que $A - B = A\cap B^c$ para quaisquer conjuntos $A$ e $B$.
    Desta forma, como a interseção entre conjuntos é associativa e comutativa temos que
    \begin{align*}
        A_m\cap A_n =& (E_m - E_{m -1}) \cap (E_n - E_{n -1})\\
        =& (E_m \cap E_{m -1}^c) \cap (E_n \cap E_{n -1}^c)\\
        =& (E_m \cap E_n) \cap ( E_{m -1}^c\cap E_{n -1}^c)\\
        =& (E_m \cap E_n) \cap \left( E_{m -1}\cup E_{n -1}\right)^c\\
    \end{align*}
    Com isso, se $m > n$, então $E_n \subseteq E_m$ e $E_{n-1} \subseteq E_{m-1}$, pois $(E_n)$ é uma sequência crescente.
    Além disso, $E_m^c \subseteq E_{m -1}^c$ e $E_m \cap E_m^c = \varnothing$.
    Segue que
    $$
    (E_m \cap E_n) \cap \left( E_{m -1}\cup E_{n -1}\right)^c =
    (E_m) \cap E_{m -1}^c = 
    \varnothing
    $$
    Caso tenhamos $m < n$ então $E_m \subseteq E_n$ e $E_{m-1} \subseteq E_{n-1}$.
    Segue analogamente que 
    $$
    (E_m \cap E_n) \cap \left( E_{m -1}\cup E_{n -1}\right)^c =
    (E_n) \cap E_{n -1}^c = 
    \varnothing
    $$
    Em todo caso, $A_m \cap A_n = \varnothing$ para todo $m \neq n$.

    Provaremos o item $(b)$ por indução sobre $n$.
    Como $(E_n)$ é crescente, temos que $E_1 \subseteq E_2$.
    Com isso, temos que 
        \begin{align*}
            \bigcup_{j = 1}^2 A_j = A_1 \cup A_2= & E_1 \cup (E_2 - E_1)\\
            = & E_1 \cup (E_2 \cap E_1^c)\\
            = & (E_1 \cup E_2) \cap (E_1 \cup E_1^c)\\
            = & (E_1 \cup E_2) \cap \mathcal{C}\\
            = & (E_1 \cap E_2)
            = E_2  
        \end{align*}
    Suponha que exista um $k \in \N$ tal que $\displaystyle \bigcup_{j = 1}^k A_j = E_k$.
    Mostraremos que $\displaystyle \bigcup_{j = 1}^{k+1} A_j = E_{k +1}$ também é verdadeira.
    Com efeito, 
    \begin{align*}
        \bigcup_{j = 1}^{k+1} A_j =& \left(\bigcup_{j = 1}^{k} A_j\right) \cup A_{k+1}\\
        =& E_k \cup A_{k +1}\\
        =& E_k \cup (E_k - E_{k+1})\\
        =& E_k \cap E_{k+1}\\
        =& E_{k+1}\\
    \end{align*}
    Segue, pelo método da indução finita, que $\displaystyle \bigcup_{j = 1}^n A_j = E_n\ \forall n \in \N$.

    Por fim, $(c)$ é um resultado imediato, pois $ x \in \displaystyle \bigcup_{j = 1}^\infty E_j$ se, e somente se, 
    existe um $n_0 \in \N$ tal que $x \in E_{n_0}$. 
    Pelo item $(b)$, isso só ocorre se $x \in \displaystyle \bigcup_{j = 1}^{n_0}A_j$.
    Mas isso é equivalente à dizer que existe um $k$ com $1\leq k\leq n_0$ tal que $x \in A_k$.
    Como $k \in \N$ isso acontece se, e somente se, $x \in A_k$ para algum $k \in \N$.
    Portanto $x \in \displaystyle \bigcup_{j = 1}^\infty A_j$.
\end{prova}

\begin{proposition}
\label{prop:sequencia-decrescente-conjuntos-resultado-A_n}
Seja $(F_n)$ uma sequência decrescente de conjuntos. 
Se $(E_n)$ é tal que $E_n = F_1 - F_n$ para todo $n \in \N$, então $(E_n)$ é crescente e 
$\displaystyle \bigcup_{j = 1}^\infty E_n = \displaystyle F_1 - \bigcup_{j = 1}^\infty F_n$.
\end{proposition}

\begin{prova}
    Queremos mostrar que $(E_n)$ é crescente, isto é, $E_n \subseteq E_{n+1}$ para todo $n \in \N$.
    Tome $x \in E_n$. Logo, $x \in F_1$ e $x \notin F_n$, por construção.
    Como $(F_n)$ é decrescente, $F_{n} \supseteq F_{n+1}$. 
    Assim, $x \notin F_n \Rightarrow x \notin F_{n+1}$.
    Com isso, $x \in F_1$ e $x \notin F_{n+1}$.
    Segue que $x \in E_{n+1}$ e que $E_n \subseteq E_{n+1}$ para qualquer $n \in \N$ como queríamos.
    Além disso, um elemento $a \in \displaystyle \bigcup_{n \in \N} E_n$ se, e somente se,
    $a \in \displaystyle \bigcup_{n \in \N} (F_1 - F_n)$.
    Isso é equivalente a dizer que existe um $n_0 \in \N$ tal que $a \in F_1 - F_{n_0}$.
    Correspondentemente, $ a \in F_1$ e existe um $n_0 \in \N$ tal que $x \notin F_{n_0}$.
    Isso só ocorre se $x \in F_1$, mas $x \notin \displaystyle \bigcup_{n \in \N} F_n$.
    Portanto, $\displaystyle \bigcup_{j = 1}^\infty E_n = \displaystyle F_1 - \bigcup_{j = 1}^\infty F_n$.
\end{prova}

\begin{definition}
\label{def:medida}
    Uma medida é uma função $\mu: (X, \mathcal{C}) \to \xreta$ tal que satisfaz as seguintes condições:
    \begin{enumerate}[label* = (\roman*)]
        \item $\mu(\varnothing) = 0$;
        \item $\mu(E) \geq 0, \ \forall A \in \mathcal{C}$;
        \item Se $(A_n)$ é uma sequência disjunta de elementos de  $\mathcal{C}$, então 
        $\displaystyle\mu\left(\bigcup_{n = 1}^\infty A_n\right) = \sum_{n = 1}^\infty\mu(A_n)$.
        
    \end{enumerate}
\end{definition}

Ou seja, uma medida é uma função não negativa que é contavelmente aditiva.
Além disso, o valor de $\mu$ pode ser igual à $+\infty$ para algum conjunto $A \in \mathcal{C}$.
Quando temos que $\mu(E) < +\infty$ para qualquer que seja o conjunto $E \in \mathcal{C}$, dizemos que temos uma medida finita.

\begin{definition}
	\label{def:espaço-de-medida}
	Dizemos que uma tripla ordenada $(X, \mathcal{C}, \mu)$ constituída por um conjunto $X$, uma \sigal $\mathcal{C}$ desse conjunto e uma medida $\mu$ sobre o espaço mensurável $(X, \mathcal{C})$ é um espaço de medida.
\end{definition}


% Exemplos de medida

\begin{example}
    Seja $X$ um conjunto e $\mathcal{C}$ a \sigal formada por todos os subconjunto de $X$.    
    Defina $\mu_1, \mu_2:\mathcal{C} \to \xreta$ pondo $\mu_1(A) = 0$ para qualquer  $A \in \mathcal{C}$ e 
    $\mu_2$ é  pondo 

$$\mu_2(A) = \left\{\begin{array}{cc}
0, & \textrm{\ se \ } A = \varnothing \\
+\infty,& \textrm{\ se \ } A \neq \varnothing
\end{array}\right.$$
Sendo definidas dessa forma, as funções $\mu_1$ e $\mu_2$ são medidas.
\end{example}

De fato, em ambas as condições \textit{(i)} e \textit{(ii)} são trivialmente satisfeitas.
Para a condição \textit{(iii)}, temos que qualquer sequência disjunta $(A_n)$ acarretará que

$$\mu_1\left(\bigcup_{n = 1}^\infty A_n\right) = 0 = \sum_{n = 1}^\infty 0 = \sum_{n = 1}^\infty \mu_1(A_n) $$
Para $\mu_2$ temos dois casos possíveis.
Se  $\displaystyle \bigcup_{n = 1}^\infty A_n  = \varnothing$, então $\mu_2\left(\displaystyle \bigcup_{n = 1}^\infty A_n\right) = 0$. Entretanto isso ocorre somente se $A_j = \varnothing$ para todo $j \in \N$.
Logo, 

$$\sum_{n = 1}^\infty \mu_2(A_n) = \sum_{n = 1}^\infty \mu_2(\varnothing) = \sum_{n = 1}^\infty 0 = 0$$
Caso $\displaystyle \bigcup_{n = 1}^\infty A_n  \neq  \varnothing$, conseguimos observar que os termos da sequência $(A_n)$ só podem ser de dois tipos ou $A_j = 0$ ou $A_j = +\infty$ para algum $j \in \N$. Com isso,  $\mu_2\left(\displaystyle \bigcup_{n = 1}^\infty A_n\right) = +\infty$.

Ademais, na soma $\displaystyle \sum_{n = 1}^\infty \mu_2(A_n)$ só teremos soma dos termos $0 + (+ \infty)$ ou $(+\infty) + (+\infty)$.
Desta forma, $\displaystyle \sum_{n = 1}^\infty \mu_2(A_n) = \sum_{n = 1}^\infty \mu_2(A_n) = +\infty$.
Portanto $\mu_1$ e $\mu_2$ são medidas.

% Probabilidade
\begin{example}[Probabilidade]
	Seja $(\Omega, \cc)$ um espaço mensurável.
	A função $\mathcal{P}:\cc \to [0,1]$ é dita uma probabilidade se satisfaz as propriedades:
	\begin{enumerate}[label* =(K\arabic*)]
		\item $\mathcal{P}(\Omega) = 1$;
		\item $\mathcal{P}(A) \geq 0,\ \forall A \in \cc$;
		\item Se $(A_n)$ é uma sequência disjunta de elementos de  $\mathcal{C}$, então 
		$\displaystyle\mathcal{P}\left(\bigcup_{n = 1}^\infty A_n\right) = \sum_{n = 1}^\infty\mathcal{P}(A_n)$.
		\footnote{As propriedades \textit{K1, K2} e \textit{K3} são chamadas de \textit{Axiomas de Kolmogorov}}
	\end{enumerate}
\end{example}

Observe que as condições da \textit{(ii)} e \textit{(iii)} da definição \ref{def:medida} são satisfeitas por definição da função de probabilidade.
Resta provar que $\mathcal{P}(\varnothing) = 0$.
Assim, com o auxilio das propriedades \textit{(K1)} e \textit{(K3)}, segue que 
$$
\pp(\Omega)
= 
\pp(\Omega \cup \varnothing)
= 
\pp(\Omega) + \pp(\varnothing)
\Rightarrow
\pp(\Omega)
= 
\pp(\Omega) + \pp(\varnothing)
$$
Logo, $\pp(\varnothing) = 0$. 
Portanto a função probabilidade é uma medida.
Neste caso, o espaço de medida $(\Omega, \cc, \pp)$ é chamado de espaço de probabilidades.
Além disso, uma função $\cc$-mensurável pela definição \ref{def:mensurabilidade-geral} em um espaço de probabilidades é chamada de variável aleatória.

% Unidade de Medida Concentrada em p
\begin{example}[Unidade de Medida Concentrada em $p$]
\label{ex:medida-concentrada-em-p}
    Seja $(X, \mathcal{C})$ um espaço mensurável onde $\cc = \mathcal{P}(X)$ e $p$ um elemento de $X$.
    Defina $\mu: \mathcal{C} \to \xreta$  como sendo


$$\mu(A) = \left\{\begin{array}{cc}
0, & \textrm{\ se \ } p \notin A \\
1, & \textrm{\ se \ } p \in A 
\end{array}\right.$$


Então $\mu$ é uma medida.
Verdadeiramente, observe que $p \notin \varnothing$, ou seja, $\mu(\varnothing) = 0$.
Trivialmente, tem-se $\mu(A) \geq 0,\ \forall A \in \mathcal{C}$, pela construção de $\mu$.



\end{example}

% Medida de contagem
\begin{example}
    Seja $X = \N$ e $\mathcal{C}$ sendo o conjunto das partes de $\N$. 
para $A \in \mathcal{C}$, definimos $\mu(A)$ por meio da sua cardinalidade, isto é, se $A$ é finito, então $\mu(A)$ é quantidade de elementos de $A$. Caso contrário, $\mu(A) = +\infty$.

\end{example}



% Teorema


\begin{theorem}
\label{teo:operacoes-com-medidas-1}
	Seja $\mu$ uma medida definida sobre uma \sigal $\mathcal{C}$.
	Se $A$ e $B$ são elementos de $\mathcal{C}$ e $A \subset B$, então $\mu(A) \leq \mu(B)$.
	Se $\mu(A) < +\infty$, então $\mu(A-B) = \mu(A) - \mu(B)$.
\end{theorem}

\begin{prova}
	Suponha que $A \subset B$, então $A = B \cup (B - A)$ e $A \cap (B - A) = \varnothing$. Segue pela propriedade $(ii)$ da definição \ref{def:medida} que 
	$$\mu(B) = \mu(A) + \mu(B-A)$$
	Lembre que $B-A = B\cap A^c$ e $A \in \mathcal{C} \Rightarrow A^c \in \mathcal{C}$.
	Além disso, como $B \in \mathcal{C}$ temos que $ B\cap A^c$ consequentemente $B - A \in \mathcal{C}$.
	Assim, como $\mu$ é uma medida e $B-A \in \mathcal{C}$, temos que $\mu(B-A) \geq 0$.
	Segue que $\mu(B) \geq \mu(A)$.
	Observe que se $\mu(A) < \infty$, temos que 

	$$\mu(B) = \mu(A) + \mu(B-A) 
 \Leftrightarrow \mu(B) - \mu(A) =  \mu(B-A)
 $$
Como desejávamos.
\end{prova}

\begin{proposition}
\label{prop:limite-sequencia-crescente}
Seja $\mu$ uma medida definida sobre uma \sigal $\mathcal{C}$.
Se $(E_n)$ é uma sequência crescente de $\mathcal{C}$, então $\mu\left(\displaystyle \bigcup_{n = 1}^\infty E_n\right) = \dlim_{n \to \infty} \mu(E_n)$.
\end{proposition} 

\begin{prova}
    Ora, se $\mu(E_n) = +\infty$ para algum $n \in \N$ ambos os lados da equação acima são $+\infty$.
    Desta forma, vamos supor que $\mu(E_n) < +\infty$ para todo $n \in \N$.
    Com isso, vamos construir uma sequência $(A_n)$ pondo $A_1 = E_1$ e $A_n = E_n - E_{n-1}$ para qualquer $n>1$.
    Então pela proposição \ref{prop:sequencia-crescente-conjuntos-resultado-A_n}, $(A_n)$ é uma sequência disjunta, temos $E_n = \bigcup_{j = 1}^n A_j$ e $\bigcup_{j = 1}^\infty E_j = \bigcup_{j = 1}^\infty A_j$.
    Como $\mu$ contavelmente aditiva, 
    $$\mu\left(\bigcup_{n = 1}^\infty E_n\right)
    =\mu\left(\bigcup_{n = 1}^\infty A_n\right)
    = \sum_{n = 1}^\infty \mu(A_n)
    = \lim_{m \to +\infty}\sum_{n = 1}^m \mu(A_n)$$
    Pelo teorema \ref{teo:operacoes-com-medidas-1} vemos que $\mu(A_n) = \mu(E_n) - \mu(E_{n - 1 })$ para $n > 1$.
    Assim, 
    
    \begin{align*}
        \lim_{m \to +\infty}\sum_{n = 1}^m \mu(A_n)
        =&
        \lim_{m \to +\infty}(\mu(A_1) + \mu(A_2) + \cdots +\mu(A_m))\\
        =&
        \lim_{m \to +\infty}(\mu(E_1) + \mu(E_2 - E_1) + \cdots +\mu(E_m - E_{m-1}))\\
        =&
        \lim_{m \to +\infty}(\mu(E_1) + \mu(E_2) - \mu(E_1) + \cdots +\mu(E_m) - \mu(E_{m-1}))\\
        =&
        \lim_{m \to +\infty}(\mu(E_1) - \mu(E_1) + \mu(E_2)  + \cdots  - \mu(E_{m-1}) +\mu(E_m) )\\
        =&
        \lim_{m \to +\infty} \mu(E_m)
    \end{align*}
    Segue que $\mu\left(\displaystyle \bigcup_{n = 1}^\infty E_n\right) = \dlim_{n \to \infty} \mu(E_n)$.




% Note que $H_2 = A_2 - A_1$ e $H_3 = A_3 - A_2$. Logo, $H_1 \cup H_2 \cup H_3$
% Assim, 

% % \begin{align*}
%     \bigcup_{n = 1}^4 H_n =&
%     \bigcup_{n = 1}^4 (A_n - A_{n-1})\\
%     =&
%     \bigcup_{n = 1}^4 \left(A_n \cap (A_{n-1})^c\right)\\
%     =&
%     \left(A_1 \cap (A_{1-1})^c\right)\\
    
% \end{align*}


\end{prova}

\begin{proposition}
Seja $\mu$ uma medida definida sobre uma \sigal $\mathcal{C}$.
Se $(B_n)$ é uma sequência decrescente de $\mathcal{C}$ e $\mu(B_1) < +\infty$, então 
$\mu\left(\displaystyle \bigcap_{n = 1}^\infty B_n\right) = \dlim_{n \to \infty} \mu(B_n)$.
\end{proposition} 
\begin{prova}
    Defina uma sequência $(T_n)$ de elementos de $\mathcal{C}$ pondo $T_n = B_1 - B_n$ para qualquer que seja $n \in \N$.
    Pela proposição \ref{prop:sequencia-decrescente-conjuntos-resultado-A_n}, $(T_n)$ é crescente.
    Assim, aplicando o a proposição \ref{prop:limite-sequencia-crescente} temos que 
    $$
    \mu\left(\bigcap_{n \in \N} T_n\right) = \lim_{n \to +\infty} \mu(T_n)
    $$
    Usando o teorema \ref{teo:operacoes-com-medidas-1}, obtemos
    $$
    \lim_{n \to +\infty} \mu(T_n) = \lim_{n \to +\infty} [\mu(B_1) - \mu(B_n)] = \mu(B_1) - \lim_{n \to +\infty} \mu(B_n)
    $$
    Segue pela proposição \ref{prop:sequencia-decrescente-conjuntos-resultado-A_n} que 
    $$
    \lim_{n \to +\infty} \mu(T_n) = \mu(B_1) - \mu\left(\bigcap_{n \in \N} B_n\right)
    $$
    Combinando as duas equações obtemos que
    \begin{align*}
        \mu(B_1) - \lim_{n \to +\infty} \mu(B_n) = \mu(B_1) - \mu\left(\bigcap_{n \in \N} B_n\right)
    \end{align*}
    Portanto, $\displaystyle \lim_{n \to +\infty} \mu(B_n) = \mu\left(\bigcap_{n \in \N} B_n\right)$
\end{prova}

% Espaços de Medida

%Uma vez que já foram bem explorados os espaços mensuráveis bem como os espaços de medida, nosso objetivo neste capítulo é medir 
